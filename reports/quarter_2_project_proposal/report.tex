% !TEX TS-program = xelatex
% !BIB TS-program = bibtex
\documentclass[12pt,letterpaper]{article}
\usepackage{style/dsc180reportstyle} % import dsc180reportstyle.sty

%%%%%%%%%%%%%%%%%%%%%%%%%%%%%%%%%%%%%%%%%%%%%%%%%%%%%%%%
%%%% Title and Authors
%%%%%%%%%%%%%%%%%%%%%%%%%%%%%%%%%%%%%%%%%%%%%%%%%%%%%%%%

\title{DSC Capstone LaTeX Template}

\author{First Author \\
  {\tt email@ucsd.edu} \\\And
  Second Author \\
  {\tt email@ucsd.edu} \\\And
  Third Author \\
  {\tt email@ucsd.edu} \\\And
  Fourth Author \\
  {\tt email@ucsd.edu} \\\And
  Mentor Name (Always Last) \\
  {\tt email@ucsd.edu} \\}

\begin{document}
\maketitle

%%%%%%%%%%%%%%%%%%%%%%%%%%%%%%%%%%%%%%%%%%%%%%%%%%%%%%%%
%%%% Abstract and Links
%%%%%%%%%%%%%%%%%%%%%%%%%%%%%%%%%%%%%%%%%%%%%%%%%%%%%%%%

{\color{blue} Welcome to the Capstone LaTeX template! To use this template:
\begin{itemize}
    \item Create an Overleaf account (it's free).
    \item Click the ``Menu" button in the top left, then click ``Copy Project."
    \item To share the project with your group members, click ``Share" in the top menu bar, turn on link sharing, and send the edit link to your group members.
\end{itemize}

The lines you see in blue are instructions from us; \textbf{remove them} or comment them out with \texttt{\%} as you actually edit this template. \textbf{Your actual submissions should not have any blue in them!}

This template mostly exists for style purposes. \textbf{Not all headings provided in the template will actually be useful for you; you can add or remove headings as you please, and the headings you end up using are up to you and your mentor.} Look to the ``Anatomy of a Scientific Paper" section of \href{https://dsc-capstone.org/lessons/04/#anatomy-of-a-scientific-paper}{Methodology Lesson 4} for guidance on which headings to use, and \href{https://dsc-capstone.org/assignments/methodology/05/}{Methodology Lesson 7} for more resources on LaTeX.
}

\begin{abstract}
    {
    \color{blue}
    Replace the \texttt{lipsum[1]} in the source code with your abstract. Don't include an abstract with your Quarter 2 Proposal.
    }
    \textcolor{LightGrey}{\lipsum[1]}
\end{abstract}

{\color{blue} For your Quarter 1 Project, you won't link a website, but will link your code. For your Quarter 2 Proposal, you won't link either of these. For your actual Quarter 2 Project, you'll link both of these.}
\begin{center}
Website: \url{https://abc.github.io/} \\
Code: \url{https://github.com/abc}
\end{center}

\maketoc
\clearpage

%%%%%%%%%%%%%%%%%%%%%%%%%%%%%%%%%%%%%%%%%%%%%%%%%%%%%%%%
%%%% Main Contents
%%%%%%%%%%%%%%%%%%%%%%%%%%%%%%%%%%%%%%%%%%%%%%%%%%%%%%%%

\section{Introduction}

\section{Methods}

\section{Results}

\section{Discussion}

\section{Conclusion}

\section{\LaTeX{} Typesetting Examples}

{\color{blue} This is not a real section; it's just here to show examples of how to format various components. Remove it before submitting!}

\subsection{\LaTeX{} Basics}

Here's \textbf{bold} and \textit{italicized} text. Here's \texttt{text\_that\_looks.like(code)}.

\begin{itemize}
    \item Here's a regular bulleted list item.
    \item And another.
\end{itemize}

Here's a \href{https://datascience.ucsd.edu}{hyperlink}. If you want to use a numbered list, you can experiment with:

\begin{enumerate}
    \item This.
    \item This.
    \item And this.
\end{enumerate}

Here's how you might include a snippet of actual code:

\begin{verbatim}
# If you want to use syntax highlighting, look into the minted package.
def f(x):
    return 2 * x + 3
\end{verbatim}

Here's how you might format a single equation:

$$\int_{-\infty}^\infty f_X(x)dx = 1$$

And a chain of equations:

\begin{align*}
    \frac{1}{n}\sum_{i = 1}^n (x_i - \bar{x})^2 &= \frac{1}{n}\sum_{i = 1}^n (x_i^2 - 2x_i\bar{x} + \bar{x}^2)
    \\ &= \frac{1}{n}\sum_{i = 1}^n x_i^2 - \frac{2}{n}\bar{x}\sum_{i = 1}^n x_i + \frac{\bar{x}^2}{n}\sum_{i = 1}^n 1
    \\ &= \frac{1}{n}\sum_{i = 1}^n x_i^2 - 2\bar{x}^2 + \bar{x}^2
    \\ &= \frac{1}{n}\sum_{i = 1}^n x_i^2 - \bar{x}^2
\end{align*}


\subsection{Figure Examples}

Here are some example figures. 
Figure \ref{fig:somefig1} presents a scatter plot.

\begin{figure}[htbp]
\centering
\includegraphics[width=.65\linewidth]{figure/somefig1.pdf}
\caption{Yes, put a few words or sentences here explaining what is in the figure.}
\label{fig:somefig1}
\end{figure}

Figure \ref{fig:someotherfigs} presents some summaries of the performance of our model.
The left panel of Figure \ref{fig:someotherfigs} presents something.
The right panel of Figure \ref{fig:someotherfigs} presents some other things.

\begin{figure}[htbp]
\begin{minipage}{0.53\linewidth}
  \centering
  \includegraphics[width=\linewidth]{figure/somefig2.png}
\end{minipage}
\begin{minipage}{0.42\linewidth}
  \centering
  \includegraphics[width=\linewidth]{figure/somefig3.png}
\end{minipage}
\caption{You can put figures side-by-side as well.}
\label{fig:someotherfigs}
\end{figure}


\subsection{Table Examples}

Table \ref{tab:sometab1} presents some summary of the data.

\begin{table}[htbp]
\caption{Some Table Caption}
\label{tab:sometab1}
\resizebox{0.4\linewidth}{!}{\input{table/sometab1}}
\end{table}

Table \ref{tab:sometab2} presents some summaries of the performance of our model.

\begin{table}[htbp]
\caption{Some Other Table Caption}
\label{tab:sometab2}
\resizebox{0.9\linewidth}{!}{\input{table/sometab2}}
\end{table}

\subsection{Equations and Algorithms Examples}

Algorithm \ref{alg:fuzzyKmeans} implements Fuzzy K-means.

\begin{algorithm}
\caption{Fuzzy K-means clustering algorithm}
\label{alg:fuzzyKmeans}
\begin{enumerate}
    \item Choose primary centroids $v_{k}$
    \item Compute the membership degree of all feature vectors in all clusters
    \begin{equation}
    u_{ki}  = \frac{1}{ \sum_{j=1}^K ( \frac{D^{2}(x_{i} - v_{k})}{D^{2}(x_{i} - v_{j})})^\frac{2} 
    {m-1}}
    \label{eq:kmeans}
    \end{equation}
\end{enumerate}
\end{algorithm}

Algorithm \ref{alg:net} calculates net activation.


\begin{algorithm}
\caption{Computing Net Activation}
\label{alg:net}
% \DontPrintSemicolon
% \LinesNumbered
\KwIn{$x_1, \ldots, x_n, w_1, \ldots, w_n$}
\KwOut{$y$, the net activation}
$y\leftarrow 0$\;
\For{$i\leftarrow 1$ \KwTo $n$}{
$y \leftarrow y + w_i*x_i$\;
}
\end{algorithm}

In Variational Autoencoder (VAE), we directly maximize the Evidence Lower Bound (ELBO) using the following Equations \ref{eq:bla}--\ref{eq:blablabla}.
\begin{align}
  \mathbb{E}_{q_{\boldsymbol{\phi}}(\boldsymbol{z}\mid\boldsymbol{x})}\left[\log\frac{p(\boldsymbol{x}, \boldsymbol{z})}{q_{\boldsymbol{\phi}}(\boldsymbol{z}\mid\boldsymbol{x})}\right]
  &= \mathbb{E}_{q_{\boldsymbol{\phi}}(\boldsymbol{z}\mid\boldsymbol{x})}\left[\log\frac{p_{\boldsymbol{\theta}}(\boldsymbol{x}\mid\boldsymbol{z})p(\boldsymbol{z})}{q_{\boldsymbol{\phi}}(\boldsymbol{z}\mid\boldsymbol{x})}\right] \label{eq:bla} \\
  &= \mathbb{E}_{q_{\boldsymbol{\phi}}(\boldsymbol{z}\mid\boldsymbol{x})}\left[\log p_{\boldsymbol{\theta}}(\boldsymbol{x}\mid\boldsymbol{z})\right] + \mathbb{E}_{q_{\boldsymbol{\phi}}(\boldsymbol{z}\mid\boldsymbol{x})}\left[\log\frac{p(\boldsymbol{z})}{q_{\boldsymbol{\phi}}(\boldsymbol{z}\mid\boldsymbol{x})}\right] \label{eq:blabla} \\
  &= \underbrace{\mathbb{E}_{q_{\boldsymbol{\phi}}(\boldsymbol{z}\mid\boldsymbol{x})}\left[\log p_{\boldsymbol{\theta}}(\boldsymbol{x}\mid\boldsymbol{z})\right]}_\text{reconstruction term} - \underbrace{\mathcal{D}_{\text{KL}}(q_{\boldsymbol{\phi}}(\boldsymbol{z}\mid\boldsymbol{x}) \mid\mid p(\boldsymbol{z}))}_\text{prior matching term} \label{eq:blablabla}
\end{align}

\subsection{Inline Citation Examples}

Citation in text (no parentheses): use \texttt{{\textbackslash}cite\{citekey\}}. 
For example, \cite{breiman2011}, \cite{devlin2019bert}.

Citation in parentheses: use \texttt{{\textbackslash}citep\{citekey\}}. 
For example: \citep{vaswani2023attention}, \citep{karras2019stylebased}.


%%%%%%%%%%%%%%%%%%%%%%%%%%%%%%%%%%%%%%%%%%%%%%%%%%%%%%%%
%%%% Reference / Bibliography
%%%%%%%%%%%%%%%%%%%%%%%%%%%%%%%%%%%%%%%%%%%%%%%%%%%%%%%%

\makereference

{\color{blue} To edit the contents of the ``References" section, edit \texttt{reference.bib}. Many conference websites format citations in BibTeX that you can copy into \texttt{reference.bib} directly; you can also search for the paper on Google Scholar, click ``Cite", and then click ``BibTeX" (\href{https://scholar.google.com/scholar?hl=en&as_sdt=0%2C23&q=attention+is+all+you+need&btnG=#d=gs_cit&t=1700436667623&u=%2Fscholar%3Fq%3Dinfo%3A5Gohgn6QFikJ%3Ascholar.google.com%2F%26output%3Dcite%26scirp%3D0%26hl%3Den}{here}'s an example).}

\bibliography{reference}
\bibliographystyle{style/dsc180bibstyle}

%%%%%%%%%%%%%%%%%%%%%%%%%%%%%%%%%%%%%%%%%%%%%%%%%%%%%%%%
%%%% Appendix
%%%%%%%%%%%%%%%%%%%%%%%%%%%%%%%%%%%%%%%%%%%%%%%%%%%%%%%%

\clearpage
\makeappendix

\subsection{Training Details}

\subsection{Additional Figures}

\subsection{Additional Tables}


\end{document}